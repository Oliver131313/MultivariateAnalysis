 \documentclass[12pt]{article}
 %\usepackage[czech]{babel}
 %\usepackage[cp1250]{inputenc}
 %\usepackage{graphicx}
 \usepackage[pdftex]{graphicx}
 \usepackage{a4wide}
 \usepackage[style=german]{csquotes}
 \usepackage{pdfpages}
 \usepackage{hyperref}

 \voffset-3cm
%\hoffset-2cm
\setlength{\textheight}{26cm}
\pagestyle{empty} %zajisti, ze stranky nebudou cislovane a s hlavickou, krome prvni stranky
\newcounter{defi}
\setcounter{defi}{0}
\renewcommand{\thedefi}{\arabic{defi}}
\newcommand{\prikl}{\stepcounter{defi}\par\medskip\noindent{\bf P��klad \thedefi}\\*[0.1ex]}
\newcommand{\nav}{\stepcounter{defi}\par\medskip\noindent{\bf \rr Instructions for the problem \thedefi:}\\*[0.1ex]}
\newcommand{\nv}{X_1,\ldots,X_n}
\newcommand{\tb}{\textbullet}
%
\newcommand{\rr}{\textcolor{blue}{\texttt{\textbf{R}}}\ }
\newcommand{\ff}{\fontfamily{cmtt}\selectfont}
\newcommand{\f}{\color{red}\fontfamily{cmtt}\selectfont}
%%%%%%%%%%%%%%%%%%%%%%%%%%%%%%%%%%%%%%%%%%%%%%%%%%%%%%%%%%%%%%%%%%%%%%%%%%%%%%%%%%%%%%%%%%%%%%%%%%%%%%%%%%%%%%%%%
%{\f }\\
\begin{document}
\begin{center}
\textbf{ \rr instructions for the 7th seminar}
\end{center}\vspace{0.3cm}
%
In a data set  \textit{Holiday.RData} there are information  about 50 families and their attitude to holiday destinations:\\
\hspace*{-2cm}\begin{tabular}{|l|l|}
\hline
 \textit{ID}& yes/no indicator showing whether a family visited  particular holiday destination during last 2 years\\
 $X_1$ & states an annual family income in thousands of dollars\\
 $X_2$ & states an attitude to travelling (on a scale 1 to 9, where 1 = absolutely negative, 9 = absolutely pozitive)\\
 $X_3$ & states an importance attributed to a family holiday (on a scale 1 to 9, where 1 =  the lowest, 9 = the highest)\\
 $X_4$ & states the number of family members\\
 $X_5$ & states an age of the oldest member of a family\\
 $V$ & states planed holiday expenditures of a family: little (1), medium (2), much (3)\\
\hline
\end{tabular}\\[1ex]
%
All following tasks do first when:\\
\textbf{a)} The \textit{ID} is a grouping variable (this must be of class factor). Explanatory variables are $X_1,\ldots,X_5$. (Data are considered to be trainig data set.)\\
At home when:\\
\textbf{b)} The \textit{V} is a grouping variable.  Explanatory variables are $X_1,\ldots,X_5$ the same.\\[1ex]
%
\nav
Get familiar with data \\
\tb Plot 2D graphs with pairs of explanatory variables and distinguish the points with regard to grouping variable. \\
{\f  plot(Holiday[,2:6],col=Holiday[,1])}\\[1ex]
\tb Calculate means of all explanatory variables separated by grouping variable; what do the results suggest?\\
{\f for(i in 2:6){cat("X",i-1,": ",tapply(Holiday[,i],Holiday[,1],mean), ' \textbackslash n')}}\\[1ex]
%%%%%%%%%%%%%%%%%%%%%%%%%%%%%%%%%%%%%%
\textsc{Canonical discriminant analysis:}\\%[1ex]
%
\nav
\tb How many canonical discriminant variables are in the model?\\
(Answer: for grouping variable = ID it is  $l=\min\{p=5,k-1=1\}=1$. Thus we are looking for just one discriminant variable.)
%
\nav
\tb Create the object in \rr bearing all essential results of Canonical discriminant analysis.
!{\f  library(candisc)}\\
{\f LinModel<-lm(cbind(X1,X2,X3,X4,X5)~ID,data=Holiday)}\\
{\f CanResults1<-candisc(LinModel,term="ID")}\\
or for groupin variable = V:\\
{\f LinModel2<-lm(cbind(X1,X2,X3,X4,X5)~V,data=Holiday)}\\
{\f CanResults2<-candisc(LinModel2,term="V")}\\
%
\nav
\tb Find out raw and standardized discriminant coefficients. (Standardized allows better interpretation.) Which variable bears the largest peace of information for discrimination?\\[1ex]
{\f CanResults1\$coeffs.raw}\\
{\f CanResults1\$coeffs.std}\\
(for standardized: $Y=-0.848X_1+\ldots+-0.469X_5$ Thus the direction of discriminant variable $Y$ is mostly influenced by $X_1$ which is an income...)
%
\nav
\tb Find out correlations between original variables and canonical variable(s). Compare its signs with signs of discriminant coefficients. Interprete it. \\[1ex]
{\f CanResults1\$structure}\\
(e.g.: $R(X_1,Y)=-0.929$). 
%
\nav
\tb Find out canonical scores. In other words express all cases by the mean of the new canonical variable(s).\\[1ex]
{\f CanResults1\$scores}\\
%
\nav 
\tb Find out means of the new canonical variable(s) for both groups.\\[1ex]
{\f CanResults1\$means}\\
("No" group mean of $Y$: 1.059855;  "Yes" group mean of $Y$: -1.463609)\\
%
\nav
\tb Express graphicaly the results of canonical discriminant analysis.\\
{\f plot(CanResults1,  fill=TRUE)}\\
{\f NewData<-data.frame(CanResults1\$scores,rep(0,times=50))}\\
{\f head(NewData)}\\
{\f plot(NewData[,2:3],col=NewData[,1])}\\[2ex]
%%%%%%%%%%%%%%%%%%%%%%%%%%%%%
for voluntiers TODO:
\textsc{Linear discriminant analysis LDA:}\\
!{\f library(MASS)}\\
%
\nav
\tb Check assumptions (multivariate normality in all groups (we wish not to reject), equality of Variance matrices (we wish not to reject), equality of mean vectors (we wish to reject)
\tb Create objects for LDA:\\
{\f LdaResults1<-lda(ID $\sim$ X1+X2+X3+X4+X5,data=Holiday,prior=c(0.5,0.5),CV=F)}\\
{\f LdaResults1}\\
To get the classification matrix:\\
{\f LdaPredict1<-predict(LdaResults1,newdata=Holiday[,2:6])\$class}\\
{\f  LdaPredict1}\\
{\f table(LdaPredict1,Holiday[,1])}\\[1ex]
%
{\f LdaResults1.1<-lda(ID $\sim$ X1+X2+X3+X4+X5,data=Holiday,prior=c(0.5,0.5),CV=T)}\\
{\f LdaResults1.1}\\
{\tiny Nefachci pro jine prior, koreny i tabulka uspesnosti stejne, jako pro 0,5}\\
{\f LdaResults2<-lda(ID $\sim$ X1+X2+X3+X4+X5,data=Holiday,prior=c(0.58,0.42),CV=F)}\\
{\f LdaResults2}\\
{\f LdaResults2.1<-lda(ID $\sim$ X1+X2+X3+X4+X5,data=Holiday,prior=c(0.58,0.42),CV=T)}\\
{\f LdaResults2.1}\\





\end{document}