 \documentclass[12pt]{article}
 %\usepackage[czech]{babel}
 %\usepackage[cp1250]{inputenc}
 %\usepackage{graphicx}
 \usepackage[pdftex]{graphicx}
 \usepackage{a4wide}
 \usepackage[style=german]{csquotes}
 \usepackage{pdfpages}
 \usepackage{hyperref}

 \voffset-3cm
%\hoffset-2cm
\setlength{\textheight}{26cm}
\pagestyle{empty} %zajisti, ze stranky nebudou cislovane a s hlavickou, krome prvni stranky
\newcounter{defi}
\setcounter{defi}{0}
\renewcommand{\thedefi}{\arabic{defi}}
\newcommand{\prikl}{\stepcounter{defi}\par\medskip\noindent{\bf P��klad \thedefi}\\*[0.1ex]}
\newcommand{\nav}{\stepcounter{defi}\par\medskip\noindent{\bf \rr Instructions for the problem \thedefi:}\\*[0.1ex]}
\newcommand{\nv}{X_1,\ldots,X_n}
\newcommand{\tb}{\textbullet}
%
\newcommand{\rr}{\textcolor{blue}{\texttt{\textbf{R}}}\ }
\newcommand{\ff}{\fontfamily{cmtt}\selectfont}
\newcommand{\f}{\color{red}\fontfamily{cmtt}\selectfont}
%%%%%%%%%%%%%%%%%%%%%%%%%%%%%%%%%%%%%%%%%%%%%%%%%%%%%%%%%%%%%%%%%%%%%%%%%%%%%%%%%%%%%%%%%%%%%%%%%%%%%%%%%%%%%%%%%
%{\f }\\
\begin{document}
\begin{center}
\textbf{ \rr instructions for the 11th seminar}
\end{center}\vspace{0.3cm}
%%%%%%%%%%%%%%%%%%%%%%%%%%%%%%%% 
Data set \textit{Emamma.RData} presents data of 1000 female patients with  breast cancer diagnosis treated in  Masaryk Oncology Institute in Brno. The list of selected variables follows:\\[1ex]
AGE � age when diagnosis was determined;\\
TIME � survival time in months;\\
Death � the status indicator, (0=alive, 1=dead);\\
SIDE � ( left and right);\\
CHT � chemotherapy (yes/no);\\
CHT\_Ttype� type of chemotherapy (no chemotherapy,  CMF,  FAC, other);\\
HT � hormonal therapy ( yes/no);\\
LR � local relapse (yes/no);\\
MTS � metastases (yes/no);\\
MP � menopause (0 - premenopausal, 1 - postmenopausal);\\
HISTOL � histology (1- ductal, 2 - lobular, 3 - modular, 4 - other);\\
STAGE � stage of tumor disease (  1, 2, 3, 4, higher values mean later stage)
%%%%%%%%%%%%%%%%%%%%%%%%%%%%%%%%%%%%%%%%%%%%%%%%%%%%%%%%%%%

\nav
For the whole data set: \\
\tb build the Kaplan-Meier estimate of survival function.\\
%
!{\f library(survival)}\\
{\f S<-Surv(Emamma\$TIME,event=Emamma\$Death)}\\
{\f SResults<-survfit(S$\sim$1,conf.type="plain", type="kaplan-meier")}\\
{\f plot(SResults,conf.int=F,xlab="survival time", ylab="survival probability")}\\[1ex]
\tb
Find out median, lower and upper quartile for the survival time.\\
{\f SResults}\\[1ex]
\tb Create confidence intervals for survival function. \\(It is based on formula: $lower=\hat{S}(t)-\sqrt{\hat{Var}(\hat{S}(t))}\cdot u_{1-\alpha/2}$;
	$\ upper=\hat{S}(t)+\sqrt{\hat{Var}(\hat{S}(t))}\cdot u_{1-\alpha/2}$)\\
{\f plot(SResults,conf.int=T,xlab="survival time", ylab="survival probability")}\\

{\f }\\
{\f }\\
{\f }\\
{\f }\\
{\f }\\

{\f }\\
{\f }\\
{\f }\\
{\f }\\
{\f }\\

{\f }\\
{\f }\\
{\f }\\
{\f }\\
{\f }\\

{\f }\\
{\f }\\
{\f }\\
{\f }\\
{\f }\\

{\f }\\
{\f }\\
{\f }\\
{\f }\\
{\f }\\


\end{document}