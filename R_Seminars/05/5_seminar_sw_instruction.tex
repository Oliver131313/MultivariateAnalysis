 \documentclass[12pt]{article}
 %\usepackage[czech]{babel}
 %\usepackage[cp1250]{inputenc}
 %\usepackage{graphicx}
 \usepackage[pdftex]{graphicx}
 \usepackage{a4wide}
 \usepackage[style=german]{csquotes}
 \usepackage{pdfpages}
 \usepackage{hyperref}

 \voffset-3cm
\setlength{\textheight}{26cm}
\pagestyle{empty} %zajisti, ze stranky nebudou cislovane a s hlavickou, krome prvni stranky
\newcounter{defi}
\setcounter{defi}{0}
\renewcommand{\thedefi}{\arabic{defi}}
\newcommand{\prikl}{\stepcounter{defi}\par\medskip\noindent{\bf P��klad \thedefi}\\*[0.1ex]}
\newcommand{\nav}{\stepcounter{defi}\par\medskip\noindent{\bf \rr Instructions for the problem \thedefi:}\\*[0.1ex]}
\newcommand{\nv}{X_1,\ldots,X_n}
\newcommand{\tb}{\textbullet}
%
\newcommand{\rr}{\textcolor{blue}{\texttt{\textbf{R}}}\ }
\newcommand{\ff}{\fontfamily{cmtt}\selectfont}
\newcommand{\f}{\color{red}\fontfamily{cmtt}\selectfont}
%%%%%%%%%%%%%%%%%%%%%%%%%%%%%%%%%%%%%%%%%%%%%%%%%%%%%%%%%%%%%%%%%%%%%%%%%%%%%%%%%%%%%%%%%%%%%%%%%%%%%%%%%%%%%%%%%
%{\f }\\
\begin{document}
\begin{center}
\textbf{ \rr instructions for the 5th seminar}
\end{center}\vspace{0.3cm}
In a data set  \textit{DistrictsCR.RData} there are some economic and demographic indicators of all districts in the Czech Republic. 
By means of the CanCor analyze association between ekonomic (variables 1 - 7) and  demographic indicators (variables 8 - 15). \\[1ex]
\begin{tabular}{|l|l|}
\hline
Earning99 & average monthly earning in 1999 in crowns\\
Unempl& measure of registered unemployment in 2000\\
Unempl2& number of unemployed per 1 job\\
Unempl3& rate of long term unemployed per all unemployed\\
BuyingPower& \\
Progre& indicator of progressivity structure in economic\\
Enterpr& number of enterprisers per 1000 inhabitants in 1999 \\
Lifex& average life expectancy in 1991-1995\\
Divorce & number of divorces per 100 marriages in 2000 \\
Abortion & number  of abortion per 100 born children\\
Pop& number of inhabitants living in small villages \\
Crime& number of crimes per 1000 inhabitants in 2000\\
Grow& relative growth of population\\
Migr& average relative migration growth 1991-2000\\
Pop65& rate of inhabitants older then 65 years in 1999\\
\hline
\end{tabular}\\[1ex]
{\f load("DistrictsCR.RData")}
%
\nav
\tb Get familiar with data (correlation matrix,scatterplot matrix,...)\\[1ex]
a)\\
b) As the first case CapitalPraha is an outlier, remove this case from the data set. The new data set entitle "Distr".\\
{\f Distr<-DistrictsCR[-1,]}\\
!{\f library(DescTools )}\\
!{\f library(ellipse)}\\
!{\f library(car)}\\[1ex]
{\f x<-cor(Distr,use="pairwise.complete.obs")}\\
{\f pairs(Distr,panel=panel.smooth)}\\
{\f scatterplotMatrix(Distr,smooth=F,diagonal="histogram",col=c(2,1,4))}\\
{\f PlotCorr(x)}\\
{\f plotcorr(x)}\\
%
\nav
\tb Is CCA appropriate method for our data? (= Is an upper corner of correlation matrix with correlations between left an 
right set variables significantly different from the unit matrix?)\\[1ex]
Visually asses graphs provided by correlation plots from previous task. 
%
\nav
\tb The matrix {\ff Distr} separate into two matrices $U$ and $V$ where $U$ represents first 7 variables associated with economic indicators and $V$ 
represents 8 variables associated with demografphic indicators.\\
{\f U<-Distr[,1:7]}\\
{\f V<-Distr[,8:15]}\\
%
%
\nav
\tb Create the object in \rr bearing all essential results of CCA. (Rather then function {\f cancor} from the package {\ff stats}
use function {\f cc } from a package {\ff CCA }\\[1ex]
!{\f install.packages("CCA")}\\
{\f Cresults<-cc(U,V)}\\
%
\nav
\tb a) Find the total redundancy for the left set of variables (resp. right set of variables.) 
In other words: What proportion of the left set variance of original variables can be explained by "right"  canonical variables?\\
\tb b) How the canonical variables represent their original variables? (How are Xs represented by Us? How are Ys represented by Vs?)\\[1ex]
See \rr script {\ff redundancy.R} 
%
\nav
\tb 
Determine the "reasonable" number of pairs of canonical variables.\\[1ex]
a) Find the values of canonical correlation coefficients\\
{\f Crho<-Cresults\$cor} notice that values in {\ff Crho} are decreasing.\\[1ex]
b) Test their significance by $\chi^2$ Bartlett test (with successive roots removed).\\
This p - values are not available in a packag CCA, for that reason another package CCP has to be installed and loaded.\\
!{\f install.packages("CCP")}\\
{\f p.asym(rho=Crho, N=76, p=7, q=8, tstat = "Wilks")}\\
(The first three canonical correlations are significant at the level 5\%.)\\[1ex]
c) Asses the scree plot of eigenvalues (see chapter 5.3. in a lecture document).\\
{\f Crho$\mbox{\^{}2}$}\\
{\f plot(Crho$\mbox{\^{}2}$ ,type="b")}\\[1ex]
d) Find redundancy for the first $k=3$ pairs of canonical variables.\\
See \rr script {\ff redundancy.R} 
%
\nav
\tb  Interpret factor structure, e.g. correlations between particular variables "from the left set" $X_i$ and canonical variable "from the left set" $U_j$. (Respectively  
correlations between particular variables "from the right set" $Y_i$ and canonical variable "form the right set" $V_j$.) 
Which variables from the left set are well represented by the first canonical variable $U_1$? \\[1ex]
{\f Cresults\$scores\$corr.X.xscores}  \\ 
e.g. $R(Earning,U_1)=0.842$ .The {\ff Earning} is well represented by $U_1$.\\
{\f Cresults\$scores\$corr.Y.yscores}\\
%
\nav
\tb 
Express cases in a new system of U and V canonical variables.(=Calculate the canonical scores.)\\[1ex]
{\f  Cresults\$scores\$xscores }\\
{\f Cresults\$scores\$yscores}\\
{\f }\\
{\f }\\
{\f }\\
{\f }\\

\end{document}