  \documentclass[12pt]{article}
 %\usepackage{czech}
 \usepackage[cp1250]{inputenc}
 \usepackage[czech]{babel}
 \usepackage{graphicx}
 \usepackage{a4wide}
 \usepackage[style=german]{csquotes}
\usepackage{pdfpages}


 \voffset-2.5cm
\setlength{\textheight}{25cm}
\pagestyle{empty} %zajisti, ze stranky nebudou cislovane a s hlavickou, krome prvni stranky
\newcounter{defi}
\setcounter{defi}{0}
\renewcommand{\thedefi}{\arabic{defi}}
\newcommand{\prikl}{\stepcounter{defi}\par\medskip\noindent{\bf Problem \thedefi}\\*[0.1ex]
           }
\newcommand{\res}{\par\medskip\noindent{\bf �e�en�}\\*[0.1ex]}
\newcommand{\nv}{X_1,\ldots,X_n}
\newcommand{\ee}{\enquote}

%%%%%%%%%%%%%%%%%%%%%%%%%%%%%%%%%%%%%%%%%%%%%%%%%%%%%%%%%%%%%%%%%%%%%%%%%%%%%%%%%%%%%%%%%%%%%%%%%%%%%%%%%%%%%%%
\begin{document}
\begin{center}
\textbf{Seminar no. 1}
\end{center}\vspace{0.3cm}
\prikl
In the data set \textit{Movies.RData}, there are 1322 respondent answers to the question: \ee{How do you assess the impact of nowadays movies on the youth?} The answers were on the 5 value scale: The impact I see as:\\
Very positive(1) - Positive(2) -  Neutral(3) -  Negative(4) -  Very negative(5).\\
Using suitable descriptive statistics characterize the data and the data categorized with regard to gender.\\
(General Social Survey, USA, 1990)
\prikl
In the data set \textit{Household\_marriage.RData}, there are 1346 respondent answers to two questions:\\
1. answer: \uv{How is it important that the youth do not live with their parents together at home?}\\
2. answer: \uv{How is it important to get married?}\\
The answers were on the 5 value scale: 
Very important(1) - Quite important(2) -  To some extent important(3) -  Not very important(4) -  Absolutely not important(5).\\
Using suitable descriptive statistics characterize both variables and compare the distribution of the two variables.\\
(General Social Survey, USA, 2002)
\prikl
Decide whether the opinion about the impact of the movies on the youth is different as far as gender is concerned in the problem 1.1. Compare and interpret $p$ values of the all possible tests. In terms of two sample $t$ test, add confidence interval of the difference between the populations' means.
\prikl
Imagine the following question regarding the problem 1.2 ($X$ variable). 
Let classify the respondents into groups of education (nominal variable $Y$) of following values:
lower than high school - high school or junior college - bachelor - graduate.\\
The data are in the \textit{Household\_education.RData} set, use the data in the \textit{fdegree4} column for the $Y$ categorical variable.\\
Detect if the distribution of the opinions about own household is different regarding particular groups of education. Compare and interpret $p$ values of all possible tests. Concerning ANOVA keep figuring out with the Tukey multi-comparison method (the Scheff� test is also possible).

\end{document}
