 \documentclass[12pt]{article}
 %\usepackage[czech]{babel}
 %\usepackage[cp1250]{inputenc}
 %\usepackage{graphicx}
 \usepackage[pdftex]{graphicx}
 \usepackage{a4wide}
 \usepackage[style=german]{csquotes}
 \usepackage{pdfpages}
 \usepackage{hyperref}
\usepackage{amssymb}
\usepackage{amsmath, amsthm}
\usepackage{bm} 


 \voffset-3cm
\setlength{\textheight}{26cm}
\pagestyle{empty} %zajisti, ze stranky nebudou cislovane a s hlavickou, krome prvni stranky
\newcounter{defi}
\setcounter{defi}{0}
\renewcommand{\thedefi}{\arabic{defi}}
\newcommand{\prikl}{\stepcounter{defi}\par\medskip\noindent{\bf P��klad \thedefi}\\*[0.1ex]}
\newcommand{\nav}{\stepcounter{defi}\par\medskip\noindent{\bf \rr Instructions for the problem \thedefi:}\\*[0.1ex]}
\newcommand{\nv}{X_1,\ldots,X_n}
\newcommand{\tb}{\textbullet}
%
\newcommand{\rr}{\textcolor{blue}{\texttt{\textbf{R}}}\ }
\newcommand{\ff}{\fontfamily{cmtt}\selectfont}
\newcommand{\f}{\color{red}\fontfamily{cmtt}\selectfont}
%%%%%%%%%%%%%%%%%%%%%%%%%%%%%%%%%%%%%%%%%%%%%%%%%%%%%%%%%%%%%%%%%%%%%%%%%%%%%%%%%%%%%%%%%%%%%%%%%%%%%%%%%%%%%%%%%
%{\f }\\
\begin{document}
\begin{center}
\textbf{ \rr instructions for the 12th seminar}
\end{center}\vspace{0.3cm}
%
# Logistic Regression
# where F is a binary factor and
# x1-x3 are continuous predictors
fit <- glm(F~x1+x2+x3,data=mydata,family=binomial())
summary(fit) # display results
confint(fit) # 95% CI for the coefficients
exp(coef(fit)) # exponentiated coefficients
exp(confint(fit)) # 95% CI for exponentiated coefficients
predict(fit, type="response") # predicted values
residuals(fit, type="deviance") # residuals \\
{\f load("GermanCredit.RData")}\\
{\f dim(GermanCredit)}\\
{\f library(DescTools)}\\
{\f WhichFactors(GermanCredit)}\\
{\f WhichNumerics(GermanCredit)}\\
\nav\vspace{-4ex}
  \begin{enumerate}\itemsep -4pt 
  \item
  \item
  \item
	 \item 
	For our data "not paying back the credit" is a success. (We are modeling "not paying back"). In \rr philosophy the first level in any factor variable is treated as "failure".
	So firstly look into data set at first three variables and check the levels of {\ff ffClass} and {\ff fClass}. \\
	{\f levels(GermanCredit\$fClass)} and {\f levels(GermanCredit\$fClass)}\\
	({\ff ffClass} is for our problem appropriate.)\\[1ex]
	%
	Create the model itself:\\
	\hspace*{-8ex}
	\framebox{\parbox{17.5cm}{{\f model<-glm(ffClass$\sim$A00Amount100,data=GermanCredit,family=binomial(link="logit"))}}}\\[1ex]
	%
	To obtain parameter estimates $\hat{\beta_k}$:\\
	{\f summary(model)}\\or simply: \\
	{\f coef(model)}\\
	$\hat{\beta_0}=-1,23$ ; $\hat{\beta_1}=0,0112$\\[2ex]
	%
	To obtain parameter estimates $e^{\hat{\beta_k}}$:\\
  {\f exp(coef(model))}\\
	%
	\item
	 To obtain confidence intervals for $\beta_k$:\\
	{\f confint(model,level = 0.95)}\\
	$\beta_0\in(-1,44\ ;\ -1,02)$  $\beta_1\in(0,0066\ ;\ 0,0158)$\\
	%
	\item
	To obtain confidence intervals for  $e^{\hat{\beta_k}}$:\\
	{\f exp(confint(model,level = 0.95))}\\
	%
	\item 
	 p-values for testing parameter's significance via Wald statistics:\\
	{\f summary(model)}\\ p-values are in the collumn "$Pr(>|z|)$".
	%
	 \item
	{\ff Null Deviance} = $D_0=-2\log L_0$, where $L_0$ is a maximum likelihood of a "null" model with nothing but an intercept. (In our case $logit(p(x_1))=\beta_0$)\\[1ex]
	{\ff Residual Deviance} = $D_1=-2\log L_1$, where $L_1$ is a maximum likelihood of a "full" model with all predictors. (In our case $logit(p(x_1))=\beta_0+\beta_1 x_1$)\\[1ex]
	Ratio $LR_{0,1}=D_0-D_1=-2\log\frac{L_0}{L_1}\approx \chi^2(df_0-df_1)$.\\
	Better model has smaller deviance. Significantly better full model then null model leads finally to large values of $LR_{0,1}$ which is considered to be a liklihood-ratio test statistic.
	Thus the concerned p-value is on the right tail of $\chi^2$ distribution.\\[1ex]
	(For our problem the likelihood-ratio test is in fact a test about a significance of the parameter $\beta_1$. 
	$LR_{0,1}=(1221.7-1199.1)=22.6$; $LR_{0,1}\approx\chi^2(999-998)$ thus the $p=0,000002)$ and the full model is significantly better then the null model. \\[1ex]
	This p-value can be obtained also by:\\
	{\f anova(model,test="Chi")}\\[1ex]
	%
  {\ff AIC}$=k-2\log L_1=k+D_1$, where $k=2*$ number of parameters. Better model has smaller $AIC$. Compared with deviance, models are penalized for large number of parameters. \\[1ex]
	(For our problem the  $AIC=2*2+1199.1=1203.1$.)
	%
   \item Hosmer-Lemeshow I dont't have\\
	{\f residuals(model, type= "deviance")} (this deviance type is also in an output of summary)\\
	(other types: "deviance", "pearson", "working","response", "partial")\\
	{\f residuals(model, type= "response")}; (this response type means: observed minus probability of success)\\
	%
	\item Fitted values:\\
	logarithmic odds ratio $\log\left( \frac{p(\bm{x})}{1-p(\bm{x})}\right)$:\\
	{\f p1<-predict(model, type="link") }\\[1ex]
	probability of success $p(\bm{x})$:\\
	{\f p2<-predict(model, type="response") } \\
  \end{enumerate}\vspace{-2ex}
%%%%
overeni predpokladu linearity prave strany:
 {\small\vspace{-2ex}
 \begin{verbatim}
 .........................................................................
gr<-rep(1:4,each=250)
dat<-data.frame(GermanCredit[,1:5])
dat<-dat[order(dat$A00Amount100),] #seradi tabulku vzestupne dle prom Amount100
dat<-data.frame(dat[,1:5],gr)
model1<-glm(ffClass~A00Amount100,data=dat[1:250,],family=binomial(link="logit"))
> model2<-glm(ffClass~A00Amount100,data=dat[250:500,],family=binomial(link="logit"))
> model3<-glm(ffClass~A00Amount100,data=dat[500:750,],family=binomial(link="logit"))
> model4<-glm(ffClass~A00Amount100,data=dat[750:1000,],family=binomial(link="logit"))$
> mean(pp1<-predict(model1, type="response"))
[1] 0.308
> mean(pp2<-predict(model2, type="response"))
[1] 0.247012
> mean(pp3<-predict(model3, type="response"))
[1] 0.2270916
> mean(pp4<-predict(model4, type="response"))
[1] 0.4183267
 .........................................................................
 \end{verbatim}}\vspace{-1ex}\noindent
Jelikoz prsti uspechu nejsou ve skupinach monotonni, nelze linearitu prave strany predpokladat

 

\end{document}