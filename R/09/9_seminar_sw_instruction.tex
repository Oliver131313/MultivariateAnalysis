 \documentclass[12pt]{article}
 %\usepackage[czech]{babel}
 %\usepackage[cp1250]{inputenc}
 %\usepackage{graphicx}
 \usepackage[pdftex]{graphicx}
 \usepackage{a4wide}
 \usepackage[style=german]{csquotes}
 \usepackage{pdfpages}
 \usepackage{hyperref}

 \voffset-3cm
%\hoffset-2cm
\setlength{\textheight}{26cm}
\pagestyle{empty} %zajisti, ze stranky nebudou cislovane a s hlavickou, krome prvni stranky
\newcounter{defi}
\setcounter{defi}{0}
\renewcommand{\thedefi}{\arabic{defi}}
\newcommand{\prikl}{\stepcounter{defi}\par\medskip\noindent{\bf P��klad \thedefi}\\*[0.1ex]}
\newcommand{\nav}{\stepcounter{defi}\par\medskip\noindent{\bf \rr Instructions for the problem \thedefi:}\\*[0.1ex]}
\newcommand{\nv}{X_1,\ldots,X_n}
\newcommand{\tb}{\textbullet}
%
\newcommand{\rr}{\textcolor{blue}{\texttt{\textbf{R}}}\ }
\newcommand{\ff}{\fontfamily{cmtt}\selectfont}
\newcommand{\f}{\color{red}\fontfamily{cmtt}\selectfont}
%%%%%%%%%%%%%%%%%%%%%%%%%%%%%%%%%%%%%%%%%%%%%%%%%%%%%%%%%%%%%%%%%%%%%%%%%%%%%%%%%%%%%%%%%%%%%%%%%%%%%%%%%%%%%%%%%
%{\f }\\
\begin{document}
\begin{center}
\textbf{ \rr instructions for the 9th seminar}
\end{center}\vspace{0.3cm}
%
\nav
\tb Get familiar with data (means and variances in groups,useful graphs). What the graphs suggest about interaction? Which factor looks to be significant?
Do variances in particular groups look similar? Do the means in particular groups look similar? Why is it important?\\
%
{\f  tapply(Crop\$Yield,list(Crop\$Irrigation,Crop\$Fertilizer),mean)}\\
{\f  tapply(Crop\$Yield,list(Crop\$Irrigation,Crop\$Fertilizer),var)}\\[1ex]
%
!{\f library(lattice)}\\
{\f xyplot(Yield$\sim$Irrigation:Fertilizer,data=Crop)}\\
{\f interaction.plot(Crop\$Irrigation,Crop\$Fertilizer,Crop\$Yield)}\\
{\f interaction.plot(Crop\$Fertilizer,Crop\$Irrigation,Crop\$Yield)}\\[1ex]
!{\f library(sciplot)}  whiskers represent standard errors for means\\
{\f lineplot.CI(Fertilizer,Yield,group=Irrigation,data=Crop)}\\
{\f lineplot.CI(Irrigation,Yield,group=Fertilizer,data=Crop)}\\
{\f lineplot.CI(Irrigation,Yield,data=Crop)}\\
{\f lineplot.CI(Fertilizer,Yield,data=Crop)}\\
%
\nav
\tb Perform a Shapiro-Wilks normality test for each group.\\
{\f tapply(Crop\$Yield,list(Crop\$Irrigation,Crop\$Fertilizer),shapiro.test)[1:6]}\\
%
\nav
\tb Perform the Levene's Test for Homogeneity of Variances.\\
!{\f  library(DescTools)}\\
{\f LeveneTest(Yield$\sim$Irrigation*Fertilizer, data=Crop,center=mean)}\\
%
\nav
\tb Create an object in \rr bearing all essential results of two-way ANOVA.\\
{\f AResults1<-lm(Yield$\sim$Irrigation+Fertilizer+Irrigation:Fertilizer,Crop)}\\
(It is vital to remember which level of Irrigation and Fertilizer is set to be reference level. In \rr it is the first one. Check it by \\
{\f levels(Crop\$Irrigation)}, the first in order is the reference one.\\
{\f levels(Crop\$Fertilizer)}\\
%
\nav
\tb Create an ANOVA table. Which factors/interactions are significant?\\
{\f anova(AResults1)}\\
%
{\small\vspace{-2ex}
\begin{verbatim}
Analysis of Variance Table

Response: Yield
                      Df Sum Sq Mean Sq  F value    Pr(>F)    
Irrigation             1    4.3    4.27   0.3582     0.552    
Fertilizer             2 4994.1 2497.07 209.6418 < 2.2e-16 ***
Irrigation:Fertilizer  2  810.1  405.07  34.0075 2.764e-10 ***
Residuals             54  643.2   11.91     
\end{verbatim}}\vspace{-1ex}\noindent
(Irrigation is insignifficant, however interaction of Irrigation with Fertilizer as well as Fertilizer  itself are significant. See the p-values in a last column.)
%
\nav
\tb Predict values in particular groups.\\
{\f tapply(predict(AResults1),list(Crop\$Irrigator,Crop\$Fertilizer),mean)}\\
{\small\vspace{-2ex}
%
\begin{verbatim}
        littleF mediumF plentyF
littleI    17.8    30.2    46.4
plentyI    21.4    37.8    36.8
\end{verbatim}}\vspace{-1ex}\noindent
%
(Remark: results are the same like descriptive means. It is because we had balanced data.)
%
\nav
\tb Create a linear model with \textsl{treatment} parametrization and interpret coefficients. (Tricky - for voluntiers!) Symbols of $\alpha$ and $\beta$ have different meaning then in lecture!!!!!\\
{\f summary(AResults1)}\\
%
{\small\vspace{-2ex}
\begin{verbatim}
Coefficients:
                                    Estimate Std. Error t value Pr(>|t|)    
(Intercept)                           17.800      1.091  16.310  < 2e-16 ***
IrrigationplentyI                      3.600      1.543   2.332   0.0234 *  
FertilizermediumF                     12.400      1.543   8.034 8.70e-11 ***
FertilizerplentyF                     28.600      1.543  18.530  < 2e-16 ***
IrrigationplentyI:FertilizermediumF    4.000      2.183   1.833   0.0724 .  
IrrigationplentyI:FertilizerplentyF  -13.200      2.183  -6.047 1.43e-07 ***
\end{verbatim}}\vspace{-1ex}\noindent
%%%%%%%%%%%%%%%%%%%%%
$\begin{array}{llcccccccccccc}
	&&&&&&&&&&&&&\\
  &   &  &\alpha&+&\beta_1.I_{pl}&+&\beta_2.F_{med}&+&\beta_3.F_{pl}&+&\beta_4.I_{pl}.F_{med}&+&\beta_5.I_{pl}.F_{pl}\\
	&&&&&&&&&&&&&\\
	\mbox{I:little} & \mbox{F:little } &| & \alpha&+&\beta_1.0&+&\beta_2.0&+&\beta_3.0&+&\beta_4.0&+&\beta_5.0\\
	\mbox{I:little} & \mbox{F:medium }& | & \alpha&+&\beta_1.0&+&\beta_2.1&+&\beta_3.0&+&\beta_4.0&+&\beta_5.0\\
	\mbox{I:little} & \mbox{F:plenty } &| & \alpha&+&\beta_1.0&+&\beta_2.0&+&\beta_3.1&+&\beta_4.0&+&\beta_5.0\\
	\mbox{I:plenty} & \mbox{F:little } &| & \alpha&+&\beta_1.1&+&\beta_2.0&+&\beta_3.0&+&\beta_4.0&+&\beta_5.0\\
	\mbox{I:plenty} & \mbox{F:medium } &| & \alpha&+&\beta_1.1&+&\beta_2.1&+&\beta_3.0&+&\beta_4.1&+&\beta_5.0\\
	\mbox{I:plenty} & \mbox{F:plenty } &| & \alpha&+&\beta_1.1&+&\beta_2.0&+&\beta_3.1&+&\beta_4.0&+&\beta_5.1\\
\end{array}$\vspace{1ex}
%
$\begin{array}{lllllll}
  	\mbox{I:little} & \mbox{F:little }  & \alpha &=&17.8&=&17.8 \\
	\mbox{I:little} & \mbox{F:medium }  & \alpha+\beta_2&=&17.8+12.4&=&30.2\\
	\mbox{I:little} & \mbox{F:plenty }  & \alpha+\beta_3&=&17.8+28.6&=&46.4\\
	\mbox{I:plenty} & \mbox{F:little }  & \alpha+\beta_1&=&17.8+3.6&=&21.4\\
	\mbox{I:plenty} & \mbox{F:medium }  & \alpha+\beta_1+\beta_2+\beta_4&=&17.8+3.6+12.4+4&=&37.8\\
	\mbox{I:plenty} & \mbox{F:plenty }  & \alpha+\beta_1+\beta_3+\beta_5&=&17.8+3.6+28.6-13.2&=&36.8\\
\end{array}$
%%%%%%%%%%%%%%%%%%%%%%
\nav
\tb Asses whether assumptions of homogeneity of variances and normality of residuals are met.\\
{\f par(mfrow=(c(1:2)))}\\
{\f plot(AResults1,which=1:2)}\\[1ex]
The first graph depicts dependence of residuals on predicted values (notice that residuals look to have homogeneous variance); the second suggests small problems with normality of residuals.\\
{\f  hist(residuals(AResults1))}\\
%
\nav
\tb  Find out the confidence intervals for population means in particular groups. To process that, first we have to perform two-way ANOVA in \textsl{textbook parametrization}. 
The easiest way to get this parametrization is to create new factor with 6 levels corresponding to 6 combinations of previous two factors. This will lead to "one way" ANOVA form without intercept. 
The new factor denote {\Nff both}\\[1ex]
{\f both<-paste(Crop\$Irrigation,Crop\$Fertilizer)}\\
{\f both<-factor(both)}\\
{\f Crop\$both<-both}\\
{\f AResults2<-lm(Yield$\sim$both-1,Crop)} "-1" is a symbol that the model is without intercept\\
{\f anova(AResults2)}\\
{\f summary(AResults2)}\\
{\small\vspace{-2ex}
\begin{verbatim}
Coefficients:
                    Estimate Std. Error t value Pr(>|t|)    
bothlittleI littleF   17.800      1.091   16.31   <2e-16 ***
bothlittleI mediumF   30.200      1.091   27.67   <2e-16 ***
bothlittleI plentyF   46.400      1.091   42.52   <2e-16 ***
bothplentyI littleF   21.400      1.091   19.61   <2e-16 ***
bothplentyI mediumF   37.800      1.091   34.63   <2e-16 ***
bothplentyI plentyF   36.800      1.091   33.72   <2e-16 ***
\end{verbatim}}\vspace{-1ex}\noindent
In the column "Estimate" there are estimated population means in particular group and the p-value in a last column has clear interpretation: Is the population mean significantly different from zero?
%
The 95\% confidence interval for any  population mean can be obtained by\\
{\f  confint(AResults2)}\\
%
{\small\vspace{-2ex}
\begin{verbatim}
                       2.5 %   97.5 %
bothlittleI littleF 15.61191 19.98809
bothlittleI mediumF 28.01191 32.38809
bothlittleI plentyF 44.21191 48.58809
bothplentyI littleF 19.21191 23.58809
bothplentyI mediumF 35.61191 39.98809
bothplentyI plentyF 34.61191 38.98809
\end{verbatim}}\vspace{-1ex}\noindent



\end{document}