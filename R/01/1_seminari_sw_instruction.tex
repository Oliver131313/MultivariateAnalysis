 \documentclass[12pt]{article}
 %\usepackage[czech]{babel}
 %\usepackage[cp1250]{inputenc}
 %\usepackage{graphicx}
 \usepackage[pdftex]{graphicx}
 \usepackage{a4wide}
 \usepackage[style=german]{csquotes}
 \usepackage{pdfpages}
 \usepackage{hyperref}

 \voffset-3cm
\setlength{\textheight}{26cm}
\pagestyle{empty} %zajisti, ze stranky nebudou cislovane a s hlavickou, krome prvni stranky
\newcounter{defi}
\setcounter{defi}{0}
\renewcommand{\thedefi}{\arabic{defi}}
\newcommand{\prikl}{\stepcounter{defi}\par\medskip\noindent{\bf P��klad \thedefi}\\*[0.1ex]}
\newcommand{\nav}{\stepcounter{defi}\par\medskip\noindent{\bf \rr Instructions for the problem \thedefi:}\\*[0.1ex]}
\newcommand{\nv}{X_1,\ldots,X_n}
\newcommand{\tb}{\textbullet}
%
\newcommand{\rr}{\textcolor{blue}{\texttt{\textbf{R}}}\ }
\newcommand{\ff}{\fontfamily{cmtt}\selectfont}
\newcommand{\f}{\color{red}\fontfamily{cmtt}\selectfont}
%%%%%%%%%%%%%%%%%%%%%%%%%%%%%%%%%%%%%%%%%%%%%%%%%%%%%%%%%%%%%%%%%%%%%%%%%%%%%%%%%%%%%%%%%%%%%%%%%%%%%%%%%%%%%%%%%
%{\f }\\
\begin{document}
\begin{center}
\textbf{ \rr instructions for the 1st seminar}
\end{center}\vspace{0.3cm}
%
\textbf{download \rr:} http://www.r-project.org/\\
\textbf{user interface for R}: http://www.rstudio.com/\\
\textbf{An R Introduction to Statistics}: http://www.r-tutor.com/\\[1.5ex]
1)set working directory, 2)check files in working directory, 3)load the data, 4)find out the names of variables, 5)check  variable classes \\*[2ex]
1) {\f setwd("C:/.../seminars/1\_seminar")}\\
2) {\f dir()}\\
3) {\f load("Movies.RData")}\\
4) {\f names(Movies)}\\
5) {\f lapply(Movies,class)}
%
\newpage
%
\nav
\tb\underline{Frequency tables}\\
absolute frequencies for the variable \textit{fMovie}: {\f table(Movies\$fMovie)} \\
absolute frequencies for all variables {\f lapply(Movies,table)}\\
relative frequencies for the variable \textit{fMovie}: {\f prop.table(table(Movies\$fMovie))} \\
relative frequencies for all variables: {\f lapply(Movies,function(my)\{prop.table(table(my))\})} \\
absolute frequencies for the variable \textit{fMovie} separately for man and woman: \newline\hspace*{2ex} {\f tapply(Movies\$fMovie,Movies\$fMan,table)  }\\[0.1ex]
relative frequencies for the variable \textit{fMovie} separately for man and woman: \newline\hspace*{2ex} {\f tapply(Movies\$fMovie,Movies\$fMan,FUN=function(my){prop.table(table(my))}) }\\[0.1ex]
%%%%%%%%%%%%%%%%%%%%%%%%%%%%%%
\tb\underline{Descriptive statistics}\\
common descriptive statistics for the variable \textit{Movie}:\\
{\f mean(Movies\$Movie, na.rm=TRUE)} \\
{\f median(Movies\$Movie, na.rm=TRUE)} \\
{\f quantile(Movies\$Movie, probs=c(0, 0.25, 0.5, 0.75, 1), na.rm=TRUE)} \\
{\f sd(Movies\$Movie, na.rm=TRUE)} \\
{\f var(Movies\$Movie, na.rm=TRUE)} \\[1.2ex]
the mean for variables \textit{Movie} and \textit{Man}:\\
{\f lapply(Movies[,1:2],mean, na.rm=TRUE)}\\[1.2ex]
common descriptive statistics for variable \textit{fMovie}:\\
{\f summary(Movies\$fMovie)}\\[1.2ex]
common descriptive statistics for all variables; notice the function {\ff summary} gives different results for numeric and factor variables:\\
{\f summary(Movies)}\\[1.2ex]
common descriptive statistics for the variable \textit{Movie} categorized by variable \textit{fMan}:\\
{\f tapply(Movies\$Movie,Movies\$fMan,summary,na.rm=T)}\\[1.5ex]
%
\framebox{\parbox{16cm}{A package "lattice" allows handy way to produce graphs like histograms, boxplots, scatterplots, etc. \textbf{Firstly}, this package should be installed;
\textbf{secondly} it should be loaded every session user intends to use commands associated with this package.\\
ad1) {\f install.packages("lattice")} ad2) {\f library(lattice)}.}}\\[1.5ex]
%%%%%%%%%%%%%%%%%%%%%%%%
!{\f library(lattice)}\\
\tb\underline{Histograms}\\
create a histogram for  variable \textit{Movie}:\\
{\f histogram(Movies\$Movie,type="count",breaks=seq(0.5,5.5,1),col=24)}\\[1.2ex]
create a histogram for  variable \textit{Movie} categorized by variable \textit{fMan}:\\
{\f histogram($\sim$Movies\$Movie\ $|$\ Movies\$fMan, type="percent",breaks=seq(0.5,5.5,1))}\\[1.2ex]
{\tiny(TODO rozmyslet oba grafy do jednoho obrazku jinak, nez pres "hist"; arg "`groups"' u histogram nefachci)}\\
Assess the skewness - can the mean be used insted of the median?\\[0.1ex]
%%%%%%%%%%%%%%%%%%%%
\tb\underline{Box plots}\\
create a boxplot for  variable \textit{Movie}:\\
{\f bwplot(Movies\$Movie)}\\[1.5ex]
create a boxplot for  variable \textit{Movie} categorized by variable \textit{fMan}:\\
{\f bwplot($\sim$Movies\$Movie|Movies\$fMan)}\\
create a boxplot for  variable \textit{Movie} categorized by variable \textit{fMan}, both graphs in one picture:\\
{\f boxplot(Movies\$Movie$\sim$Movies\$fMan)}\\
%%%%%%%%%%%%% 
\nav
...analogously, e.g.:\\
{\f summary(Household.marriage)}\\
{\f lapply(Household.marriage,function(my)\{m=mean(my);s=sd(my);v=var(my);return(c(m,s,v))\})}\\[1.5ex]
{\f histogram(Household.marriage\$getmar ,type = "percent")}\\
{\f histogram(Household.marriage\$ownhh ,type = "percent")}\\[1.5ex]
 Notice that the shape of histogram  of variable \textit{ownhh} is skewed positively whereas \textit{getmar} is "symetric". Thus using of the mean in case of the
 \textit{ownhh} is unacceptable whereas in case of the \textit{getmar} it can be accepted.\\
 {\tiny(TODO rozmyslet oba grafy do jednoho obrazku, par(mfrow(c(1,2)) nefachci}\\
%%%%%%%%%%%%%%
 \nav
 \tb\underline{$t$-test}\\
 There are 1322 cases in a sample, so according to the large sample size, $t$-test is acceptable.\\
Have a look at categorized histograms, descriptive statistics,... \\
{\f histogram($\sim$Movies\$Movie| Movies\$fMan)}\\
{\f bwplot($\sim$Movies\$Movie| Movies\$fMan)}\\
{\f tapply(Movies\$Movie,Movies\$fMan,function(my)\{m=mean(my);v=var(my);return(c(m,v))\})}\\[1.5ex]
Before running t-test it is necessary to assess the assumption of equal sigmas which is done via F-test:\\
{\f var.test(Movies\$Movie[Movies\$fMan=="man"],Movies\$Movie[Movies\$fMan=="woman"],ratio=1,\\
alternative="two.sided")}\\[1.5ex]
As the F-test did not rejected equality of sigmas (p-value = 0.4058) we can proceed with two-sample t-test:\\
{\f t.test(Movies\$Movie[Movies\$fMan=="man"],Movies\$Movie[Movies\$fMan=="woman"],\\mu=0,var.equal=T)}\\[1.5ex]
(Notice, p-value of the t-test (p-value = 0.0001746) can be supplemented by  \textit{Cohen's d} for effect size. Code for calculating \textit{Cohen's d} is at the end of this file.)\\
%%%%%%%%%%%%%%%%%%%%%%%%%%%%%%%%%%%%%%%%%
 \tb\underline{Wilcoxon rank sum test}\\
A) Exact test\\
!{\f library(exactRankTests)}\\
{\f wilcox.exact(Movies\$Movie[Movies\$fMan=="man"],Movies\$Movie[Movies\$fMan=="woman"],mu=0,exact=TRUE)}\\
This exact test is feasiable only for small sample sizes, so considered sample "Movies" can not be processed by this test.\\ [1.5ex]
%
B) Asymptotical test without continuity correction\\
{\f wilcox.test(Movies\$Movie[Movies\$fMan=="man"],Movies\$Movie[Movies\$fMan=="woman"],mu=0,\\exact=FALSE, correct=FALSE)}\\[1.5ex]
%
C) Asymptotical test with continuity correction.\\
{\f wilcox.test(Movies\$Movie[Movies\$fMan=="man"],Movies\$Movie[Movies\$fMan=="woman"],mu=0,\\exact=FALSE, correct=TRUE)}\\
%%%%%%%%%%%%%%%%%%%%%
\tb\underline{$\chi^2$ test}\\
Firstly we have to create  contingency tables of categorial variables fMovie and FMan:\\
{\f t<-table(Movies\$fMovie,Movies\$fMan); t} table of absolute frequences\\
{\f addmargins(t)} absolute frequences with margins\\
{\f prop.table(t) } relative frequences (cell percentages)\\
{\f addmargins(prop.table(t))} relative frequences with margins\\
{\f  prop.table(t,1)} row percentages\\
{\f  prop.table(t,2)} column percentages\\
{\f  margin.table(t,1)} row margins \\
{\f  margin.table(t,2)}  column margins\\[1.5ex]
Performing $\chi^2$-test:\\
{\f chisq.test(t)} provides p-value and the test-statistic\\
{\f chisq.test(t)\$expected}\\
{\f chisq.test(t)\$observed}\\
{\f chisq.test(t)\$residuals}\\
%%%%%%%%%%%%%%%%%%%%%
\nav
\tb\underline{ANOVA}\\
 As there are 1322 values in a data set ANOVA is acceptable.
 \begin{itemize}
   %
	\item[a)] \textbf{Categorised boxplots in one picture: }\\
	{\f boxplot(Household.education\$ownhh$\sim$Household.education\$fdegree4)}\\
	%
	\item[b)] \textbf{Descriptive statistics: }\\
	{\f tapply(Household.education\$ownhh,Household.education\$fdegree4,mean)}\\
	{\f tapply(Household.education\$ownhh,Household.education\$fdegree4,sd)}\\
	{\f tapply(Household.education\$ownhh,Household.education\$fdegree4,summary)}\\
	 \item[c)] \textbf{Categorized histograms in separate pictures: }\\
	{\f histogram($\sim$Household.education\$ownhh|Household.education\$fdegree4)}\\ 	
	(Categorized QQ-plots can be supplemented. A code for it is at the end of this ?le.)
	%
	\item[d)] \textbf{Assumption of equality of variances:} \\
	!{\f library(DescTools)}\\
	{\f LeveneTest(Household.education\$ownhh$\sim$Household.education\$fdegree4,center=mean)}\\[0.5ex]
	(For our data equality of variances was rejected and ANOVA should not be performed. Following steps are just to demonstrate the \rr functions related to ANOVA method. 
	However, \rr offers also function for performing ANOVA with unequal variances, 	this test is 	only asymptotical and not included in basic textbooks.\\
	{\f oneway.test(Household.education\$ownhh$\sim$Household.education\$fdegree4,var.equal = FALSE)})
	%
	\item[e)] \textbf{ANOVA test:}\\
	{\f model<-lm(Household.education\$ownhh$\sim$Household.education\$fdegree4)}\\
  Attention! The factor variable in a model definition must be of class "factor". Otherwise, e.g class "numeric" for variable "degree" will lead to false results.\\
	{\f anova(model)}\\
	or
	{\f aov(Household.education\$ownhh$\sim$Household.education\$fdegree4)}\\
	{\f summary(aov(Household.education\$ownhh$\sim$Household.education\$fdegree4))}\\
  %
	\item[f)] \textbf{Post-hoc tests:} \\
	Firstly package "agricolae" has to be installed and downloaded.\\	
	!{\f library(agricolae)}\\
	{\f scheffe.test(aov(ownhh$\sim$fdegree4,data=Household.education),"fdegree4", group=TRUE,\\ console=TRUE,alpha=0.104)}\\
	(This function does not provide p-values, only shows whether or not means are significantly different at the $\alpha$ level. When  $\alpha=0.05$, no pair is for our data significantly different; here contrast can be significant.)\\[1ex]
	TukeyHSD test is appropriate for balanced design (which is not our case), thus following \rr function is just to demonstrate a syntax.\\
	{\f TukeyHSD(aov(Household.education\$ownhh$\sim$Household.education\$fdegree4))}\\
	(Performed p-values are smaller then true p-values, as the test assumes balanced groups.)\\[1ex]
	Another option in base package is Bonferroni Multipl comparisons method:\\
	{\f pairwise.t.test(Household.education\$ownhh,Household.education\$fdegree4,\\ p.adjust.method="bonferroni") }\\
	  %
	\item[f)] \textbf{Analyzing residuals:} \\
	{\f shapiro.test(residuals(lm(Household.education\$ownhh$\sim$Household.education\$fdegree4)))}
	(As the sample size is large it is no surprise that the normality was rejected.\\[1ex]
	{\f qqnorm(residuals(lm(Household.education\$ownhh$\sim$Household.education\$fdegree4)))}\\
	{\f qqline(residuals(lm(Household.education\$ownhh$\sim$Household.education\$fdegree4)))}\\
\end{itemize}
  %
 \tb\underline{Kruskal-Walis test}\\
  {\f histogram($\sim$Household.education\$ownhh|Household.education\$fdegree4, type="percent")}\\
  {\f kruskal.test(Household.education\$ownhh$\sim$Household.education\$fdegree4)} \\[1ex]
	%
   \tb\underline{$\chi^2$ test}\\
	{\f t<-table(Household.education\$ownhh , Household.education\$fdegree4); t} \\
	{\f chisq.test(t)}\\[2ex]
\textbf{	script for Cohen's $d$:}
	{\small\vspace{-2ex}
	%%%%%%%%%%%%%%%%%%%% script for Cohen
		\begin{verbatim}
	.........................................................................
		cohens_d <- function(x, y) {
    lx <- length(x)- 1
    ly <- length(y)- 1
    md  <- abs(mean(x) - mean(y))        ## mean difference (numerator)
    csd <- lx * var(x) + ly * var(y)
    csd <- csd/(lx + ly)
    csd <- sqrt(csd)                     ## common sd computation

    cd  <- md/csd                        ## cohen's d
}
> res <- cohens_d(Movies$Movie[Movies$fMan=="man"],Movies$Movie[Movies$fMan=="woman"])
> res
.........................................................................
	\end{verbatim}}\vspace{0ex}\noindent
	%%%%%%%%%%%
	\textbf{script for normal Q-Q plot:}
	{\small\vspace{-2ex}
	\begin{verbatim}
	.........................................................................
	cat.qq<-function(x,y){
  l<-length(levels(y))
  par(mfrow=c(l/2,2))
  for(i in levels(y) ){qqnorm(x[y==i],main=(paste(i," Normal Q-Q Plot")) )  
                       qqline(x[y==i])}
  }
	.........................................................................
	\end{verbatim}}\vspace{-1ex}
	useful:\\
!{\f library(DescTools)}\\
{\f Freq(Household.marriage\$ownhh )	}\\
{\f PercTable(Household.marriage\$ownhh , Household.marriage\$getmar)}\\
\end{document}
